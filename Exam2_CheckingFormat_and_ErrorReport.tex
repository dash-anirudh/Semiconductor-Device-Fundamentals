
\documentclass[12pt]{article}\date{}

\usepackage{geometry}
\geometry{left=0.5in, right=0.5in, top=0.5in, bottom=0.75in}

\usepackage{changepage}

\usepackage{graphicx}

\usepackage{amsmath}

\usepackage{bm}

\begin{document}

\title{Checking format and Common Errors}
\maketitle

\noindent Checking list: \\
Q1-2: Om Priya \\
Q3-4:  Piyush \\
Q5-6: Sourav \\
Q7-13: Manaswini \\
Q14, Q16: Nikhil \\
Q15, Q17:  Anirudh \\
Q18: Nishat \\ \\

\noindent Totalling 1: Nishat \\
Totalling 2, Rechecking: Anirudh \\
Evaluation report: Anirudh \\

\noindent Evaluation deadline: $28^{th}$ Nov. \\ 
Evaluation completion: $25^{th}$ Nov. \\
Evaluation report completion: $3^{rd}$ Dec. \\

\noindent Min: 3 \\
Max: 45.5 \\
Mean:  27.45 \\
Median: 27.50 \\
Standard Deviation: 9.16 \\

\section{Section 1}
\subsection{Q1}
The question involves the direct application of a formula taught in class. Most students have solved it correctly. Even if the final answer is not in eV, full marks have been awarded.
\subsection{Q2}
The question involves the direct application of a formula taught in class. Most students have solved it correctly. 
\subsection{Q3}
Some students have used an incorrect circuit diagram for the CE configuration. Errors have been found in correctly identifying and bifurcating the emitter and collector sections of the network and correctly applying the power supply. All students with the correct circuit diagram have gone on to solve the question correctly, bar some minor calculation mistakes, for which 0.5 marks have been deducted. If a calculation error has been made in the early stages of solving, but the procedure is correct, only 1 mark has been deducted.
\subsection{Q4}
Most students have solved this question correctly. Many students have not approximated $2.5 \times 10^{13}+10^{17}$ as $10^{17}$, and have found precise answers. Full marks have been awarded for the same.
\subsection{Q5}
Several students have not accounted for $I_{CBO}$ whilst solving, leading to an incorrect result. Apart from this, many students have solved this correctly.
\subsection{Q6}
Several students have failed to correctly identify which diodes will be forward-biased and which ones are reverse-biased. Post this; many students have not accounted for the effective voltage as $Vin-2\times 0.7$, where $0.7$ is the barrier potential. Finally, some students haven't found $R_{eff}$ correctly.

\section{Section 2}
For Q8, no marks have been given for "current source"; only "constant current source" or its equivalent has been given marks. For Q11, all equivalent terms for "heavily doped", "thin/ very thin", and "high/ very high" have been awarded full marks. 

\section{Section 3}
[Step marking has been done up to the last theoretically correct step. If there are any calculation mistakes leading to errors cascading down, marks have still been awarded]
\subsection{Q14}
Some students have not used $V_{BE}=0.7$ for solving the question and have thus found the data insufficient. Apart from this, there are only minor calculation errors, and most students have solved the question correctly.
\subsection{Q15}
Some students have not found the Thevenin voltage and resistance. Many students have not solved the simultaneous equations involving emitter and collector current correctly. There are minor errors while applying KVL and choosing loops as well. Marks have been given for mentioning critical formulas involving the above parameters and those involving amplification factors. All resistance values between $280 \Omega$ and $300 \Omega$ for emitter resistances and $580 \Omega$ and $610 \Omega$ have been given marks if everything before this is found to be correct.
\subsection{Q16}
Some students have taken the signs for $V_{GS}$ and $V_{GS, \ cutoff}$ incorrectly, leading to wrong answers. Apart from that, most solutions are correct. Even if the operating point hasn't been explicitly mentioned in the form of (current, voltage) or (voltage, current), marks have been given.
\subsection{Q17}
Some students have used a graphical approach to solve the question, while others have solved it using predetermined formulas. All valid and conceptually correct methods have been awarded full marks for the right result. Many students have found the voltage of the operating point correctly but have only calculated the corresponding collector current after that instead of the base current. 

\subsection{Q18}
Most students who have attempted the bonus question have solved it correctly.

\end{document}