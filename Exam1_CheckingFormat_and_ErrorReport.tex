\documentclass[12pt]{article}\date{}

\usepackage{geometry}
\geometry{left=0.5in, right=0.5in, top=0.5in, bottom=0.75in}

\usepackage{changepage}

\usepackage{graphicx}

\usepackage{amsmath}

\usepackage{bm}

\begin{document}

\title{Checking format and Common Errors}
\maketitle

\noindent Checking list: \\
Q1-7: Om Priya \\
Q8-14: Manaswini \\
Q15-16: Anirudh \\
Q17-18: Nikhil \\
Q19: Sourav \\
Q20: Piyush \\
Q21: Nishat \\ \\
Totalling: Anirudh \\
Presentation of solutions to students and doubt-clearing session: Anirudh \\
Evaluation report: Anirudh \\

\noindent Evaluation deadline: $5^{th}$ Oct. \\ 
Evaluation completion: $3^{rd}$ Oct. \\
Evaluation report completion: $7^{th}$ Oct. \\

\section{Section 1}
Most of the solutions are straightforward, and a majority of students have solved this set of questions correctly. \\ The only question where a significant chunk of students has faltered is $Q8$. The concept to be used was the number of electrons that exit the valence band create an equal number of holes in the conduction band. Thus, $\Delta n= \Delta p$. However, most students have solved this incorrectly.
\section{Section 2}
For $Q11$, several students have written electrons or charge carriers as the answer instead of electron-hole pairs. "Charge carriers" has been considered as correct. However, no marks have been awarded for "electrons" as an answer. \\
For $Q13$, some students have written "Current" as the answer. No marks have been given for the same.

\section{Section 3}

\subsection{Q15}

Error 1: Identifying the correct body diagonal plane. Any body diagonal plane of a cube must pass through 4 vertices- which are 2 pairs of opposite vertices. It should contain both diagonals involving those pairs of vertices. 

\begin{figure}[!ht]
  \centering
  \includegraphics[width=0.75\textwidth]{BodyDiagonals.drawio}
\begin{tiny}
\caption{Body Diagonal Planes}
\end{tiny}
\end{figure}

\noindent Only the shown diagrams have been accepted. A very common error has been the use of the following diagram:

\begin{figure}[!ht]
  \centering
  \includegraphics[width=0.60\textwidth, height=0.40\textheight]{IncorrectBodyDiagonalPlane.drawio}
\begin{tiny}
\caption{Incorrect Body Diagonal Plane}
\end{tiny}
\end{figure}

\noindent This does not satisfy the basic definition of a body diagonal plane of a cube. When extended in all directions, the plane does not pass through 2 pairs of opposite vertices. \\
\\
Error 2: Some students have also drawn a similar diagram with intercepts as (a, a/2, a) or (2a, a, 2a). Both of these fail for the same reason. \\
\\
Error 3: For students who have got 2 indices correct and the third one incorrect, 2 marks have been awarded. \\
\\
Error 4: For those who haven't labelled the axes but have the correct indices, 0.5 marks have been deducted. \\ \\
For students who have solved multiple body diagonal planes, 0.5 marks have been cut for each wrong set of Miller Indices. Some students have also used the following body diagonal:

\begin{figure}[!ht]
  \centering
  \includegraphics[width=0.75\textwidth]{IncorrectMillerIndices.drawio}
\begin{tiny}
\caption{Alternate Choice of Body Diagonal Plane}
\end{tiny}
\end{figure}

\noindent Error 5: Here, the x and z intercepts are 0, and the y-intercept is undefined but finite. This should give the correct result; however, students have failed to identify the intercepts and, thus, the indices correctly. 0.5 marks have been given in this case as well. \\
\\
To all students who mentioned anything about Wiese Indices, reciprocals, etc., with the incorrect Miller indices, 0.5 marks have been awarded. \\

\subsection{Q16}

\noindent Full marks have been given to all students who have written the formula correctly and found the correct answer, even if they haven't mentioned that $N_{A}$ is approximately equal to $n_{p}$. \\ \\
Error 1: Several students have written the unit of electron concentration as atoms per $m^3$. In all such cases, I have deducted 0.5 marks. \\
\\
Error 2: Some students have directly added concentrations as follows: $(1+0.25) \times 10^{20}$ and then calculated the result as ${6.25 \times 10^{38} \over 1.25 \times 10^{20}}$, which gives $5 \times 10^{18}$ instead of $6.25 \times 10^{18}$. 2 marks have been given in these cases. \\ \\
Some students have solved using quadratic equations, i.e., they have found the exact values using $p_0+N_d-n_0-N_a=0$. When used along with $np=n^2_i$, this gives a result of $5.9 \times 10^{18}$. Full marks have been awarded in all such cases. However, if there is some calculation error, 0.5 marks have been cut. \\

\subsection{Q17}
\noindent This question involves the direct application of the formula given in class, and most students have solved it correctly. Some students have used an incorrect formula, depending on whether or not they believed that charge 'q' of the electron was to be included in the formula or whether it had been subsumed during the derivation of the formula itself.
\subsection{Q18}
\noindent Most students have solved this correctly.
\subsection{Q19}
\noindent Most students have solved this correctly. Some students have substituted the value of kT and calculated the final numerical answer as well, despite the fact that it was mentioned in the question to solve in terms of kT. No marks have been deducted for the same.
\subsection{Q20}
\noindent Not more than 20 students have solved this question correctly. \\
Error 1: Incorrect direction on the basis of change in potential at the heterojunction. \\ \\
Error 2: Not accounting for the fall in potential (either overcompensating or missing it altogether) \\ \\
Error 3: Incorrect identification of Fermi Level \\ \\
Full marks have been awarded if the entire diagram is correct. 1 mark has been awarded if only either the top or bottom section of the heterojunction band diagram is correct. 2 marks have been given if everything is correct except for the Fermi Level labelling.
\subsection{Q21}
\noindent Most students have solved this correctly. Some students have missed out on considering the effective masses, leading to an incorrect result. Partial marks have been awarded up to the last correct step (0.5 per step). 
\end{document}