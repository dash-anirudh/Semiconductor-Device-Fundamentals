\documentclass[12pt]{article}\date{}

\usepackage{geometry}
\geometry{left=0.5in, right=0.5in, top=0.5in, bottom=0.75in}

\usepackage{changepage}

\usepackage{graphicx}

\usepackage{amsmath}

\usepackage{bm}

\begin{document}
\title{Mid Semester Examination}
\maketitle
\begin{center}
\begin{large}
\author{EE2104: Semiconductor Device Fundamentals} \\[30pt]
\end{large}
\end{center} 
\textbf{Time Allotted}: 55 minutes \\
\textbf{Total Marks}: 32 \\
\textbf{Bonus Question}: 21- 3 marks [So you can score 35/32] \\[30pt]

\begin{large}
\textbf{SECTION 1: MULTIPLE CHOICE QUESTIONS (1 MARK EACH)} \\[30pt] 
\end{large}

\noindent 1. In a P type material, the Fermi level is 0.3 eV above the valence band. The concentration of acceptor atoms is increased. The new position of Fermi level is likely to be \\
a. 0.5 eV above the valence band. \\
b. 0.2 eV above the valence band. \\
c. Below the valence band. \\
d. None of the above \\

\noindent 2. The drift velocity of electrons in silicon \\
(a) is proportional to the electric field for all values of the electric field \\
(b) is independent of the electric field \\
(c) increases at low values of electric field and decreases at high values of electric field exhibiting negative differential resistance \\
(d) increases linearly with the electric field and gradually saturates at higher values of the electric field \\

\noindent 3. The type of recombination which takes place via an extra energy level is called \\
a. Radiative recombination \\
b. Auger recombination \\
c. Shockley-Read-Hall recombination \\
d. Surface recombination \\

\noindent 4. At zero K (or at absolute zero), the conduction band may be partially filled in \\
(a) Conductors only \\
(b) Insulators only \\
(c) Semiconductors only \\
(d) Conductors and semiconductors \\

\noindent 5. Donor impurity atoms in semiconducting material result in a new \\
(a) Wide energy band \\
(b) Narrow energy band \\
(c) Discrete energy level just below conduction band \\
(d) Discrete energy level just above valance band \\

\noindent 6. What is the atomic radius of the BCC crystal structure? \\
a) a/2 \\
b) a/4 \\
c) a $\sqrt2$/4 \\
d) a $\sqrt3$/4 \\

\noindent 7. If the temperature of an intrinsic semiconductor is increased so that the intrinsic carrier concentration is doubled, then: \\
a) The majority carrier density doubles \\
b) The minority carrier density doubles \\
c) Both majority and minority carrier densities double \\
d) None of the above \\

\noindent 8. Due to illumination by light, the electron and hole concentrations in a heavily doped N type semiconductor increase by $\Delta$ n and $\Delta$ p, respectively. If $n_i$ is the intrinsic concentration, then \\
a) $\Delta$ n $<$ $\Delta$ p \\
b) $\Delta$ n $>$ $\Delta$ p \\
c) $\Delta$ n $=$ $\Delta$ p \\
d)  None of the above \\

\noindent 9. Which one of the following breakdowns occurs in the thin region? \\
a) Avalanche \\
b) Zener \\
c) Both a and b \\
d) None of the above \\

\noindent 10. In a triclinic crustal, a lattice plane makes intercepts at a length a, 2b and -3c/2. The Miller indices of the plane are: \\
a) 3:6:4 \\
b) 6:3:4 \\
c) 6:3:-4 \\
d) 6:3:-2 \\ [30pt]

\begin{large}
\textbf{SECTION 2: FILL IN THE BLANKS (1 MARK EACH)} \\[30pt]
\end{large} 

\noindent 11. In a semiconductor, current conduction is due to \rule{1cm}{0.15mm}  \\
12. A semiconductor is formed by \rule{1cm}{0.15mm} bonds \\
13. The equation $J_n$ $=qnµ_nE$ (A/cm 2 ) represents \rule{1cm}{0.15mm}\\
14. In a doped semiconductor, the \rule{1cm}{0.15mm} carriers have a lesser lifetime minority \\[30pt]

\begin{large}
\textbf{SECTION 3: NUMERICAL QUESTIONS (3 MARK EACH)} \\[30pt]
\end{large} 

\noindent 15. Calculate the Miller indices for any of the body diagonal planes of a cube. \\ [15pt]

\noindent 16. A p-type semiconductor has an acceptor density of $10^{20}$ atoms/$m^3$ and an intrinsic concentration of $2.5 \times 10^{19}/m^3$ at 300K. The electron concentration (per $m^3$) in this p-type semiconductor is \\ [15pt]

\noindent 17. Mobilities of electrons and holes in a sample of intrinsic semiconductors at room temperature are $0.36 m^2$/Vs and $0.17 m^2$/Vs, respectively. If both electrons and hole densities in semiconductor equal $2.5 \times 10^{19}/m^3$, then the conductivity of the semiconducting material is \\ [15pt]

\noindent 18. Calculate the diffusion current density for a given semiconductor: consider silicon at T=300 K. Assume electron concentration varies linearly from n = $10^{12}/cm^3$ to n = $10^{16}/cm^3$ over a distance from x = 0 to x = 3 um. Assume $D_n = 35 cm^2$/s. \\ [15pt]

\noindent 19. Calculate the energy related to the Fermi energy for which the Fermi function equals 5 per cent. Write the answer in units of kT. \\ [15pt]

\noindent 20. Draw the band diagram of the hetero junction shown below, assuming MgZnO to be heavily doped in comparison to CdZnO. \\ [15pt]

\begin{figure}[!ht]
  \centering
  \includegraphics[width=0.75\textwidth]{BandDiagram.drawio}
\end{figure}

\noindent 21 [Bonus]. Derive an expression for the Fermi level for an intrinsic semiconductor \\ [15pt]


\end{document}